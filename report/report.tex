\documentclass{article}

\usepackage{amsmath, amssymb, amsthm}
\usepackage{geometry}
\usepackage{tcolorbox}
\usepackage{fancyvrb}
\usepackage{listings}
\usepackage[hidelinks]{hyperref}
\usepackage{tikz, wasysym, tikz-cd}
\usepackage{mathtools} %xrightarrow
\usetikzlibrary{automata,arrows.meta,positioning,calc,decorations.pathmorphing}
\usepackage{etoolbox} % setcounter{counter}{num}
\usepackage{enumitem} % \begin{enumerate}[label=(\alph*)]

\renewcommand{\baselinestretch}{1.5}

% QUIVER STUFF
% A TikZ style for curved arrows of a fixed height, due to AndréC.
\tikzset{curve/.style={settings={#1},to path={(\tikztostart)
    .. controls ($(\tikztostart)!\pv{pos}!(\tikztotarget)!\pv{height}!270:(\tikztotarget)$)
    and ($(\tikztostart)!1-\pv{pos}!(\tikztotarget)!\pv{height}!270:(\tikztotarget)$)
    .. (\tikztotarget)\tikztonodes}},
    settings/.code={\tikzset{quiver/.cd,#1}
        \def\pv##1{\pgfkeysvalueof{/tikz/quiver/##1}}},
    quiver/.cd,pos/.initial=0.35,height/.initial=0}
% TikZ arrowhead/tail styles.
\tikzset{tail reversed/.code={\pgfsetarrowsstart{tikzcd to}}}
\tikzset{2tail/.code={\pgfsetarrowsstart{Implies[reversed]}}}
\tikzset{2tail reversed/.code={\pgfsetarrowsstart{Implies}}}
% TikZ arrow styles.
\tikzset{no body/.style={/tikz/dash pattern=on 0 off 1mm}}

\theoremstyle{plain}
\newtheorem{theorem}{Theorem}
\newtheorem{lemma}{Lemma}
\newtheorem{example}{Example}

\theoremstyle{definition}
\newtheorem{definition}{Definition}
\newtheorem{problem}{Problem}

\newenvironment{solution}
  {\renewcommand\qedsymbol{$\blacksquare$}\begin{proof}[Solution]}
  {\end{proof}}

\lstset{
    emph={
        IntegrateAssignNames,
        DefiningPolynomial,
        ConstantField,
        Parent,
        PolynomialRing,
        RationalField,
        SquarefreeFactorisation,
        SquarefreePartialFractionDecomposition,
        XGCD,
        Derivative,
        hom,
        Roots,
        UnivariatePolynomial,
        Resultant,
        LogarithmicFieldExtension,
        OrderedGenerators,
        UnderlyingField,
        DifferentialFieldExtension,
        ConstantFieldExtension,
        IsAlgebraicDifferentialField,
        Generators,
        CoefficientRing,
        ExponentialFieldExtension,
    },
    emphstyle=\color{violet},
    emph={[2]
        eq,
        in,
        "&+"
    },
    emphstyle={[2]\color{teal}},
    numbers=left,
    basicstyle=\ttfamily,
    numberstyle=\tiny,
    breaklines=true,
    xleftmargin=.25in,
    escapechar='
}

\DeclareMathOperator{\im}{im}
\DeclareMathOperator{\dom}{dom}
\DeclareMathOperator{\cod}{cod}
\DeclareMathOperator{\id}{id}
\DeclareMathOperator{\res}{res}
\DeclareMathOperator{\lc}{lc}
\newcommand{\angles}[1]{\left<#1\right>}
\newcommand{\catname}[1]{{\normalfont\textbf{#1}}}
\newcommand{\GL}{\text{GL}}
\newcommand{\SL}{\text{SL}}
\newcommand{\sg}{\text{sg}}
\newcommand{\dd}{\mathop{}\!d}
\newcommand{\x}{{\bf x}}
\newcommand{\y}{{\bf y}}
\newcommand\defeq{\stackrel{\mathclap{\normalfont\mbox{\scriptsize def}}}{=}}
\newcommand{\N}{\mathbb{N}}
\newcommand{\Z}{\mathbb{Z}}
\newcommand{\Q}{\mathbb{Q}}
\newcommand{\R}{\mathbb{R}}
\newcommand{\C}{\mathbb{C}}

\newcommand{\exref}[1]{\emph{Example \ref{#1}}}
\newcommand{\defref}[1]{\emph{Definition \ref{#1}}}
\newcommand{\propref}[1]{\emph{Proposition \ref{#1}}}
\newcommand{\thmref}[1]{\emph{Theorem \ref{#1}}}
\newcommand{\lineref}[1]{Line \ref{#1}}

\DeclarePairedDelimiterX\set[1]\lbrace\rbrace{\def\given{\;\delimsize\vert\;}#1}
 
\newcommand{\naturalto}{%
  \mathrel{\vbox{\offinterlineskip
    \mathsurround=0pt
    \ialign{\hfil##\hfil\cr
      \normalfont\scalebox{1.2}{.}\cr
%      \noalign{\kern-.05ex}
      $\longrightarrow$\cr}
  }}%
}

\title{Magma Symbolic Integration Report}
\author{Mitchell Holt}

\begin{document}

\maketitle

\tableofcontents

\newpage

\section{Introduction and Background}

This report describes an incomplete implementation of the Risch procedure for
symbolic integration, as described by Geddes \cite{geddes:afca}. Rational
integration (see DEFINITION TODO) has been implemented completely. Logarithmic
integration (see DEFINITION TODO) is implemented, but not correct. The tests
contain some failing cases. Although exponential integration is partially
described in Geddes \cite{geddes:afca}, it has not been implemented. The
integration of algebraic functions or other transcendentals (such as error
functions) was beyond the scope of the project. \medbreak

A significant design decision in the code is to allow the \emph{Risch Structure
Theorem} \cite{risch:algprops} to be used to verify that each new logarithmic or
exponential elementary field extension is transcendental, as is required. The
\emph{Risch Structure Theorem} has not been implemented, but if one could be
created then it would be trivial to add it to the existing code.

\section{Differential Fields}

In this section we describe the intrinsics in \lstinline{DifferentialFields.m},
which are motivated by \defref{elementary_field}.

\begin{definition}[Elementary Function Field] \label{elementary_field}
    Let $K = \mathbb{Q}(\alpha_1, \dots, \alpha_m)$ be a constant differential
    field (with differential operator $'$) where each $\alpha_i$ is algebraic
    over $\mathbb{Q}$. Let $x$ be a transcendental symbol over $K$ satisfying
    $x' = 1$ (in the differential field $(K(x),\, ')$). In the differential
    extension field $K(x, \theta_1, \dots, \theta_n)$, we say each $\theta_j$
    ($j = 1, \dots, n$) is \emph{elementary} over $K(x, \theta_1, \dots,
    \theta_{j - 1})$ if $\theta_j$ is \emph{transcendental} over $K(x, \theta_1,
    \dots, \theta_{j - 1})$ and either:
    \begin{enumerate}[label=(\roman*)]
        \item $\theta_j' = u'/u$ for some $u \in K(x, \theta_1, \dots,
            \theta_{j - 1})$ (in which case we say $\theta_j$ is
            \emph{logarithmic}), or
        \item $\theta_j'/\theta_j = u'$ for some $u \in K(x, \theta_1, \dots,
            \theta_{j - 1})$ (in which case we say $\theta_j$ is
            \emph{exponential}).
    \end{enumerate}
    If each $\theta_j$ is elementary, then we say that $K(x, \theta_1, \dots,
    \theta_{j - 1})$ is an \emph{elementary function field}.
\end{definition}

Ideally, we would represent the elementary function field $K(x,\, \theta_1,\,
\dots,\, \theta_m)$ in Magma as an \lstinline{RngDiff} whose constant field is
either the rational field or an absolute number field, and whose (ordered)
generators are $x,\, \theta_1,\, \dots,\, \theta_m$. However, there are
elementary function fields which cannot be constructed with a single call to the
existing \lstinline{DifferentialFieldExtension} intrinsic because it requires
the universe of the input sequence to be a multivariate polynomial ring (in our
case over $K(x)$), rather than the field of fractions. For example, consider the
elementary function field $\Q(x,\, \log x,\, \log(\log x))$, where
$\log(\log x)$ has derivative $\frac 1 {x \log x} \not \in \Q(x)[\log x,\,
\log(\log x)]$. \medbreak

Therefore we represent elementary function fields either as $K(x)$, or as
$F(\theta)$, where $F$ is an elementary function field and $\theta$ is
elementary over $F$. In the case that $F$ is not $K(x)$, using
\lstinline{DifferentialFieldExtension} does not set the generators of the field
it returns correctly. Nils Bruin provides a fix for this bug in the case of a
logarithmic extension:

\begin{lstlisting}[numbers=none]
fld := LogarithmicFieldExtension(F, f); // incorrect generators
fld`Generators := [ fld | c : c in OrderedGenerators(UnderlyingField(fld)) ];
\end{lstlisting}

This does not work for exponential extensions. The intrinsic
\lstinline{IsLogarithmic} assumes that the contruction of exponential extensions 
works as intended (although it currently does not).

\subsection{Documentation}

\subsubsection*{\lstinline{IsPolyFractionField(F) : RngDiff -> BoolElt}}
Checks if \lstinline{F} is of the form $K(x)$ for some constant field $K$.

\subsubsection*{\lstinline{AsFraction(f) : RngDiffElt -> FldFunRatElt}}
Return the representation of \lstinline{f} as a fraction inside the underlying
field of \lstinline{Parent(f)}.

\subsubsection*{\lstinline{IsPolynomial(f) : RngDiffElt -> BoolElt, RngUPolElt}}
Let \lstinline{Parent(f)} be either $K(x)$ for some constant field $K$ or
$F(\theta)$ for some elementary function field $F$ with $\theta$ elementary over
$F$. Check if \lstinline{f} is in $K[x]$ or $F[\theta]$ respectively, and return
its representation as a univariate polynomial if it is.

\subsubsection*{\lstinline{ExtendConstantField(F, C) : RngDiff, Fld -> RngDiff}}
This is the same as \lstinline{ConstantFieldExtension}, but works for any
elementary function field (not just $K(x)$, which satisfies
\lstinline{IsAlgebraicDifferentialField}).


\subsubsection*{\lstinline{IsLogarithmic(F) : RngDiff -> BoolElt}}
Check if \lstinline{F} is of the form $F(L)$ for some elementary function field
$F$ and elementary function $L$.

\subsubsection*{\lstinline{AllLogarithms(F) : RngDiff -> SeqEnum}}
Give an ordered sequence of all of the logarithmic generators of \lstinline{F},
with the last logarithm in the extension tower being the last to appear in the
sequence.

\subsubsection*{\lstinline{IsTranscendentalLogarithm(new, logarithms) : RngDiffElt, SeqEnum -> SeqEnum}}
Let \lstinline{logarithms} be the logarithmic generators of an elementary
function field $F$ and \lstinline{new} be of the form $\frac {u'} u$ for some
$u \in F$. Use the \emph{Risch Structure Theorem} to check if $\log u$ is
transcendental over $F$. If it is, return the empty sequence. If not, return a
sequence $S$ of \lstinline{< factor, non-zero power >} pairs such that $u =
\prod_{(g, k) \in S} g^k$.

\subsubsection*{\lstinline{LogarithmicExtension(F, f) : RngDiff, RngDiffElt -> RngDiff, SeqEnum, RngDiffElt}}

Named parameters:
\begin{itemize}
    \item[] \lstinline{logarithms : SeqEnum} with default value \lstinline{[]}.
\end{itemize}

\noindent Return the differential extension field $F(L)$, where $L$ is
logarithmic over $F$ with derivative $f$, noting that it may be the case the
$F(L) = F$. Also return all of the logarithms of $F(L)$ and the representation
of $L$ inside of $F(L)$. If \lstinline{logarithms} is non-empty, it is assumed
that these are all of the logarithmic generators of $F$. Otherwise, the
logarithms will be calculated.

\subsubsection*{\lstinline{ExponentialExtension(F, f) : RngDiff, RngDiffElt  -> RngDiff, SeqEnum, RngDiffElt}}

Named parameters:
\begin{itemize}
    \item[] \lstinline{exponentials : SeqEnum} with default value \lstinline{[]}.
\end{itemize}

\noindent Return the differential extension field $F(E)$, where $E$ is
exponential over $F$ with derivative $fE$, noting that it may be the case the
$F(E) = F$. Also return all of the logarithms of $F(E)$ and the representation
of $E$ inside of $F(E)$. If \lstinline{exponentials} is non-empty, it is assumed
that these are all of the exponential generators of $F$. Otherwise, the
exponentals will be calculated. \medbreak

This function does not currently work, for example the derivative of $\exp x$ is
not set correctly in the construction of $\Q(x,\, \log x,\, \exp x)$. It seems
that the problem may be in the intrinsic \lstinline{DifferentialFieldExtension},
which is called by \lstinline{ExponentialFieldExtension}.

\lstinputlisting{exp_err}

\subsubsection*{\lstinline{NameField(~F) : RngDiff ->}}

Named parameters:
\begin{itemize}
    \item[] \lstinline{FirstExtName : MonStgElt} with default value
        \lstinline{"x"}.
    \item[] \lstinline{AlgNumName : MonStgElt} with default value
        \lstinline{"a"}.
\end{itemize}

\noindent Using sensible defaults, name the transcendental generators of $F$ and
the generator of its constant field $K$ (if any) for readable printing of
elements of $F$. \lstinline{FirstExtName} is the name to be assigned to the
(unique) transcendental over the constant field which has derivative $1$.

\section{Integration}
There is a single intrinsic inside \lstinline{Integration.m}, which handles
deciding how to integrate an elementary function and calls the relevant routine,
depending on whether the input is a rational function, is logarithmic, or is
exponential.

\begin{definition}[Integration] \label{elt_int}
    Let $F$ be an elementary function field and $f \in F$. If there exists a
    finitely and explicitly generated elementary extension $G$ of $F$ and a $g
    \in G$ with $g' = f$, then we say $g$ is the \emph{elementary
    anti-derivative} or \emph{elementary integral} of $f$ and write $\int f =
    g$. If no such $g$ or $G$ exist, we say that $f$ has no elementary integral.
\end{definition}

\subsection{Documentation}

\subsubsection*{\lstinline{ElementaryIntegral(f) : RngDiffElt -> BoolElt, RngDiffElt, SeqEnum}}

Named parameters:
\begin{itemize}
    \item[] \lstinline{all_logarithms} with default value \lstinline{[]}.
\end{itemize}

\noindent Check if the given elementary function has an elementary integral. If
it does, return an integral and a list of all the logarithms needed to generate
the smallest extension of the field that the input comes from. \medbreak

Note that, by \emph{Lioville's Principle} (see Geddes \cite{geddes:afca}, \S
12.4) the only elementary extensions required to express the integral of an 
elementary function are logarithmic, so we need only return all of the
logarithmic generators of the field the integral comes from.

\section{Rational Integration}
A \emph{rational function} is an element of $\Q(x)$. Note that, for rational
integration, an elementary integral always exists.

\subsection{Documentation}

\section{Logarithmic Integration}
\begin{itemize}
    \item Brief intro.
    \item Cite Geddes.
    \item Give derivations of the less obvious parts of some intrinsics.
\end{itemize}

\subsection{Documentation}

\addcontentsline{toc}{section}{References}
\bibliographystyle{plain}
\bibliography{bibliography}

\end{document}
